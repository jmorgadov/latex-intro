\section{Estructurar un documento}
\frame{\tableofcontents[currentsection]}

\subsection{Estilo}
\begin{frame}[fragile]{Estilo}
La clase del documento (\texttt{documentclass}) define el estilo del
mismo y además varios comandos. En el primer ejemplo se mostró la clase
\texttt{article}, sin embargo existen muchas otras: \texttt{proc},
\texttt{book}, \texttt{report}, \texttt{letter}, etc.
\end{frame}

\subsection{Presentación}
\begin{frame}[fragile]{Presentación}

Una página de presentación se puede estrucuturar manualmente:

\begin{center}
\begin{columns}

\begin{column}{0.5\textwidth}
\begin{lstlisting}[basicstyle=\tiny]
\begin{center}
  \text{\Large My project title}\\
  \text{\small John Doe}\\
  \vspace{0.5cm}
  \text{\tiny Jul 14th, 2023}\\
\end{center}
\end{lstlisting}
\end{column}

\begin{column}{0.5\textwidth}
\begin{center}
  \text{\Large My project title}\\
  \text{\small John Doe}\\
  \vspace{0.5cm}
  \text{\tiny Jul 14th, 2023}\\
\end{center}
\end{column}

\end{columns}
\end{center}

\pause

Sin embargo, la gran mayoria de las clases que tiene \LaTeX por defecto
brindan comandos para estructurar de manera sencilla la presentación de un
documento:

\begin{lstlisting}[basicstyle=\tiny]
\title{My project title}
\author{John Doe}
\date{Jul 14th, 2023}
\maketitle
\end{lstlisting}

\end{frame}

\subsection{Secciones, subsecciones y más...}
\begin{frame}[fragile]{Secciones, subsecciones, y más...}
Específicamente en la clase \texttt{article} se pueden definir secciones,
subsecciones y subsubsecciones. Existen varias ventaja que trae organizar el
documento usando estos comandos: enumeración automática, generación de tablas
de contenido de forma automática, son fáciles de referenciar a lo largo del
documeto, etc.

\begin{lstlisting}[basicstyle=\tiny]
\section{Introduction}
...
\section{Main topic}
...
\subsection{A sub-topic}
...
\section{Other topic}
...
\end{lstlisting}

\end{frame}
