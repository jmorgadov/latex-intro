\section{Introducción}\label{intro}

\begin{frame}
    \begin{columns}[t]
        \begin{column}{.5\textwidth}
          \tableofcontents[sections={1-2},currentsection]
        \end{column}
        \begin{column}{.5\textwidth}
          \tableofcontents[sections={3-4},currentsection]
        \end{column}
    \end{columns}
\end{frame}

\subsection{¿Qué es \LaTeX?}

\begin{frame}{¿Qué es \LaTeX?}
\begin{center}
Un software para crear documentos
\end{center}

\pause

Creado por \textbf{Leslie Lamport} en 1984 con el objetivo de extender \TeX, el
cual a su vez fue creado por \textbf{Donald Knuth}.

\end{frame}


\subsection{Ventajas y desventajas}
\begin{frame}{Ventajas y desventajas}

Ventajas:
\begin{itemize}
  \item Es gratis
  \item Produce documentos de alta calidad
  \item Fácil control de versiones
  \item Diseñado específicamente para renderizar de una mejor forma ecuaciones,
    figuras, tablas etc
  \item Independiende de plataforma
\end{itemize}

\pause

Desventajas:
\begin{itemize}
  \item Errores de compilación poco explicativos
  \item Procesos de compilación complicados (generalmente los maneja el editor
    de texto)
\end{itemize}

\end{frame}

\subsection{Instalación}

\begin{frame}{¿Cómo instalar latex?}

\begin{block}{Linux}
  Usar el gestor de paquetes según la distribución. Por ejemplo:\\[2mm]
  \textbf{Ubuntu}: \texttt{apt install texlive-latex-extra}\\
  \textbf{Arch}: \texttt{pacman -S texlive-latexextra}\\
\end{block}

\pause

\begin{block}{MacOS}
  \textbf{Básico}: \texttt{brew install --cask basictex}\\
  \textbf{Completo}: \texttt{brew install --cask mactex}\\
\end{block}

\pause

\begin{block}{Windows}
  Pueden descargar \emph{MiKTex} desde: \url{https://miktex.org/download}
\end{block}
\end{frame}

\subsection{Editor de texto}

\begin{frame}{Editor de texto}

Editores específicos:

\begin{itemize}
  \item TexStudio
  \item TeXShop
  \item TexMaker
  \item TexWork
  \item ...muchos otros
\end{itemize}

\pause

Editores generales que también pueden usar:

\begin{itemize}
  \item Visual Studio Code
  \item Sublime Text
  \item Vim
  \item ...básicamente cualquier cosa que edite texto
\end{itemize}

\end{frame}

