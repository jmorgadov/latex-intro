\section{Nociones básicas}

\begin{frame}
    \begin{columns}[t]
        \begin{column}{.5\textwidth}
          \tableofcontents[sections={1-2},currentsection]
        \end{column}
        \begin{column}{.5\textwidth}
          \tableofcontents[sections={3-4},currentsection]
        \end{column}
    \end{columns}
\end{frame}

\subsection{Estructura del documento}
\begin{frame}[fragile]{Estructura del documento}

Lo principal que hay que declarar es el tipo de documento y el cuerpo del
mismo:

\begin{lstlisting}
\documentclass{article}

\begin{document}
 Hello world
\end{document}
\end{lstlisting}

\pause
\vspace{5mm}
Existen dos conceptos principales en \LaTeX: comandos y entornos.

\end{frame}

\subsection{Comandos y entornos}
\begin{frame}[fragile]{Comandos}

Un comando ejecuta una acción. La misma puede ser para establecer una
configuración, importar algun paquete, o incluso cambiar la forma en la que se
renderiza un texto. Los comandos están compuestos de la siguiente forma:

\begin{verbatim}
\nombre[opciones]{parámetro}
\end{verbatim}

\pause

Ejemplos:\\[2mm]
- \textbackslash\texttt{LaTeX} ~produce~ \LaTeX\\
- \textbackslash\texttt{textbf\{foo\}} ~produce~ \textbf{foo}\\
- \textbackslash\texttt{documentclass[a4paper]\{article\}} Declara el tipo de documento como artículo usando tamaño de hoja A4.

\end{frame}

\begin{frame}[fragile]{Entornos}

Los entornos se usan para aplicar un formato o configuración específica a una
parte del documento. Los entornos se estructuran de la siguiente forma:

\begin{lstlisting}
\begin{nombre}[opciones]{parametro}
 % Contenido
\end{nombre}
\end{lstlisting}

\end{frame}

\begin{frame}{Ejemplos de comandos}

\begin{center}
\begin{tabular}{lr}
  \texttt{\textbackslash textbf\{Hello world\}} & \textbf{Hello world} \\[3mm]
  \texttt{\textbackslash textit\{Hello world\}} & \textit{Hello world} \\[3mm]
  \texttt{\textbackslash emph\{Hello world\}} & \emph{Hello world} \\[3mm]
  \texttt{\textbackslash texttt\{Hello world\}} & \texttt{Hello world} \\[3mm]
  \texttt{\textbackslash underline\{Hello world\}} & \underline{Hello world} \\[3mm]
\end{tabular}
\end{center}

\end{frame}

\begin{frame}{Ejemplos de comandos}
\begin{center}
\color{red}\text{\small\emph{Estos comandos afectan al entorno completo donde se
encuentren}}\color{black}\\[5mm]

\begin{tabular}{lr}
  \texttt{\textbackslash tiny Hello world} & \tiny Hello world \\
  \texttt{\textbackslash small Hello world} & \small Hello world \\
  \texttt{\textbackslash large Hello world} & \large Hello world \\
  \texttt{\textbackslash Large Hello world} & \Large Hello world \\
  \texttt{\textbackslash huge Hello world} & \huge Hello world \\
  \texttt{\textbackslash Huge Hello world} & \Huge Hello world
\end{tabular}
\end{center}
\end{frame}

\subsection{Ecuaciones y tablas}
\begin{frame}[fragile]{Ecuaciones}
Existen 2 formas principales de declarar expresiones matemáticas:
\begin{itemize}
  \item Utilizando el caracter \$\\
    Ejemplo: \texttt{\$x=\textbackslash sqrt\{3\}\$} ~produce~ $x=\sqrt{3}$\\[1mm]
    \text{\small \emph{Se usa para agregar expresiones dentro de un texto}}
  \item Utilizando entornos matemáticos\\
    Ejemplo:
    \begin{lstlisting}
    \begin{equation}
    f(x) = \sqrt{\frac{\pi}{x}}
    \end{equation}
    \end{lstlisting}
    produce:
    \begin{equation}
    f(x) = \sqrt{\frac{\pi}{x}}
    \end{equation}
\end{itemize}
\end{frame}

\begin{frame}[fragile]{Tablas}

Las tablas se escriben usando el entorno \texttt{tabular}. Analicemos el
siguiente ejemplo:

\begin{lstlisting}
\begin{center}
\begin{tabular}{l|ccr}
  Nombre & Edad & Sexo & Provincia \\
  \hline
  Juan & 26 & M & Cienfuegos\\
  Ana & 25 & F & La Habana\\
\end{tabular}
\end{center}
\end{lstlisting}

Esto produce la siguiente tabla:

\begin{center}
\begin{tabular}{l|ccr}
  Nombre & Edad & Sexo & Provincia \\
  \hline
  Juan & 26 & M & Cienfuegos\\
  Ana & 25 & F & La Habana\\
\end{tabular}
\end{center}

\end{frame}

\begin{frame}[fragile]{Tablas}

\begin{lstlisting}
\begin{center}
\begin{tabular}{l|ccr}
  Nombre & Edad & Sexo & Provincia \\
  \hline
  Juan & 26 & M & Cienfuegos\\
  Ana & 25 & F & La Habana\\
\end{tabular}
\end{center}
\end{lstlisting}

\only<1>{
El entorno \texttt{tabular} recibe como parámetro la configuración de las
columnas. En este caso (\texttt{l$|$ccr}) es: una columna alineada la izquierda
(\texttt{l}), una división ($|$), dos columnas centradas (\texttt{cc}) y una
columna alineada a la derecha (\texttt{r}).
}
\only<2>{
Las filas se escriben en el cuerpo del entorno, separando cada columna entre \&
y cada fila entre \textbackslash\textbackslash. Además se pueden utilizar comandos para
agregar separadores, unir celdas, etc.
}

\end{frame}

\begin{frame}{Tablas}
Las tablas se pueden customizar tanto como se quiera:

\begin{center}
\scalebox{.45}{
\begin{tabular}{ccccccccccccc}
    \toprule
    \multicolumn{2}{c}{\multirow{2}[4]{*}{}} &
    \multicolumn{10}{c}{\bf Actual} &
    \multirow{2}[4]{*}{\bf Precision} \\
    & \cline{2-11}
    & & 0 & 1 & 2 & 3 & 4 & 5 & 6 & 7 & 8 & 9 & \\
    \midrule
    \multirow{10}{*}{\bf Predicted}
        & 0 & \textbf{75.71 \%} & 0.0 \% & 1.26 \% & 0.79 \% & 1.32 \% & 3.36 \% & 5.64 \% & 0.1 \% & 18.07 \% & 1.39 \% & 70.6 \% \\
        & 1 & 0.0 \% & \textbf{97.71 \%} & 0.1 \% & 0.3 \% & 0.2 \% & 0.34 \% & 0.52 \% & 1.56 \% & 0.1 \% & 0.2 \% & 97.11 \% \\
        & 2 & 0.31 \% & 0.0 \% & \textbf{53.0 \%} & 6.73 \% & 4.18 \% & 1.68 \% & 3.34 \% & 3.31 \% & 4.21 \% & 0.2 \% & 69.86 \% \\
        & 3 & 0.2 \% & 0.18 \% & 8.24 \% & \textbf{59.7 \%} & 8.55 \% & 2.24 \% & 5.22 \% & 3.79 \% & 4.93 \% & 0.4 \% & 64.35 \% \\
        & 4 & 3.47 \% & 0.44 \% & 10.37 \% & 14.85 \% & \textbf{63.03 \%} & 6.84 \% & 10.33 \% & 6.71 \% & 3.18 \% & 1.68 \% & 51.93 \% \\
        & 5 & 3.37 \% & 0.26 \% & 1.36 \% & 1.29 \% & 2.14 \% & \textbf{72.09 \%} & 5.53 \% & 2.33 \% & 5.65 \% & 18.73 \% & 61.35 \% \\
        & 6 & 9.29 \% & 0.0 \% & 3.1 \% & 3.86 \% & 9.88 \% & 5.61 \% & \textbf{55.64 \%} & 1.17 \% & 3.29 \% & 1.98 \% & 58.83 \% \\
        & 7 & 0.0 \% & 1.41 \% & 13.86 \% & 9.5 \% & 6.72 \% & 2.8 \% & 7.93 \% & \textbf{79.18 \%} & 0.62 \% & 5.05 \% & 62.95 \% \\
        & 8 & 7.04 \% & 0.0 \% & 6.2 \% & 2.67 \% & 0.81 \% & 2.35 \% & 4.49 \% & 0.1 \% & \textbf{56.67 \%} & 1.59 \% & 68.91 \% \\
        & 9 & 0.61 \% & 0.0 \% & 2.52 \% & 0.3 \% & 3.16 \% & 2.69 \% & 1.36 \% & 1.75 \% & 3.29 \% & \textbf{68.78 \%} & 81.94 \% \\
        \midrule
    \multicolumn{2}{c}{\bf Recall} & 75.71 \% & 97.71 \% & 53.0 \% & 59.7 \% & 63.03 \% & 72.09 \% & 55.64 \% & 79.18 \% & 56.67 \% & 68.78 \% \\
    \bottomrule
\end{tabular}
}
\end{center}


\end{frame}

\subsection{Imágenes}
\begin{frame}[fragile]{Imágenes}

Mostrar una imagen se puede lograr usando el paquete \texttt{graphicx}. Por ejemplo:

\begin{lstlisting}
\usepackage{graphicx} % fuera del entorno `document`
...
\begin{center}
\includegraphics[width=3cm]{matcom.jpg}
\end{center}
\end{lstlisting}

...produce:

\begin{center}
\includegraphics[width=3cm]{matcom.jpg}
\end{center}

\end{frame}
